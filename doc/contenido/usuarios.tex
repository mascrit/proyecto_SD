\documentclass[../main.tex]{subfiles}

\begin{document}

Se debe ejecutar el script \texttt{add\_user.sh},
en este ejeplo añadiremos al un usuario nombrado como
\textit{usuario\_77}.

\begin{listing}[H]
\begin{minted}{shell-session}
$ ./add_user.sh usuario_77
\end{minted}
\end{listing}

Si todo sale bien se le pedira la contraseña de UNIX y
la de SAMBA (Usar la misma).

\begin{minted}{shell-session}
Añadiendo el usuario 'usuario_77' ...
make: se entra en el directorio '/var/yp'
make[1]: se entra en el directorio '/var/yp/srv.nis'
Updating netid.byname...
make[1]: se sale del directorio '/var/yp/srv.nis'
make: se sale del directorio '/var/yp'
Añadiendo el nuevo grupo 'usuario_77' (1010) ...
make: se entra en el directorio '/var/yp'
make[1]: se entra en el directorio '/var/yp/srv.nis'
Updating group.byname...
Updating group.bygid...
Updating netid.byname...
make[1]: se sale del directorio '/var/yp/srv.nis'
make: se sale del directorio '/var/yp'
Añadiendo el nuevo usuario 'usuario_77' (1010) con grupo 'usuario_77' ...
make: se entra en el directorio '/var/yp'
make[1]: se entra en el directorio '/var/yp/srv.nis'
Updating passwd.byname...
Updating passwd.byuid...
Updating netid.byname...
Updating shadow.byname...
make[1]: se sale del directorio '/var/yp/srv.nis'
make: se sale del directorio '/var/yp'
Creando el directorio personal '/home/usuario_77' ...
Copiando los ficheros desde '/etc/skel' ...
Nueva contraseña:
Vuelva a escribir la nueva contraseña:
passwd: contraseña actualizada correctamente
Cambiando la información de usuario para usuario_77
Introduzca el nuevo valor, o pulse INTRO para usar el valor predeterminado
        Nombre completo []: Usuario 77
        Número de habitación []: 12b
        Teléfono del trabajo []: 5567382132
        Teléfono de casa []: 5536271232
        Otro []:
¿Es correcta la información? [S/n] S
Ingresa la contraseña SAMBA del usuario
New Password:
Retype Password:
User 'usuario_77' created successfully
make[1]: se entra en el directorio '/var/yp/srv.nis'
Updating passwd.byname...
Updating passwd.byuid...
Updating netid.byname...
Updating shadow.byname...
make[1]: se sale del directorio '/var/yp/srv.nis'
\end{minted}

El contenido de \texttt{add\_user.sh} es el siguiente:
\begin{listing}[H]
\inputminted{bash}{../configs/add_user.sh}
\caption{Contenido de add\_user.sh}
\label{listing:adduser.sh}
\end{listing}

\end{document}
