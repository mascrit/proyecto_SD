\documentclass[../main.tex]{subfiles}

\begin{document}
\subsection{Red Emulada}\label{sec:red_emu}

\begin{table}[H]
  \centering
  \begin{tabular}{rl}
    
    Segmento:&$192.168.100.0/24$\\
    Puerta de enlace:&$192.168.100.1$\\
    Broadcast:&$192.168.100.255$\\
    Dominio:&\lstinline|srv.nis|\\
  \end{tabular}
\end{table}

\subsection{Servidor}\label{sec:servidor}

\subsubsection{Servidor Linux (VM)}\label{sec:slvm}


\begin{table}[H]
  \centering
  \begin{tabular}{rl}
    
    Hostname: &Node03\\
    Sistema Operativo: & Debian 10 \textit{Buster}\\
  \end{tabular}
\end{table}

\subsubsection{Tarjeta de Red}\label{sec:tr}

\begin{table}[H]
  \centering
  \begin{tabular}{rl}
    
    IP:&$192.16.100.119/24$\\
    DNS:&$192.168.100.119 8.8.8.8$\\
  \end{tabular}
\end{table}

\subsection{Configuración}\label{sec:serv_conf}

\subsubsection{Configurar la tarjeta de red}\label{sec:conf_tr}

Se tiene que configurar la tarjeta de red para que adquiera su DNS y
ip estática:

\begin{itemize}
\item En este caso la interfaz de red es \lstinline|ens33|, donde
  este nombre puede variar.
\item Se tiene que modificar el archivo \lstinline|/etc/network/interfaces|
  y añadir la siguiente configuración:
\lstset{style=nonestyle}
  \begin{lstlisting}[label={list:interfaces},caption=Archivo /etc/netctl/interfaces]
auto ens33
allow-hotplug ens33
iface ens33 inet static
    address 192.168.100.119
    netmask 255.255.255.0
    network 192.168.100.0
    broadcast 102.168.100.255
    gateway 192.168.100.1
    dns-nameservers 192.168.100.119 8.8.8.8
    dns-search srv.nis
  \end{lstlisting}
\end{itemize}

\paragraph{Asignar Dominio}\ \\
Se debe de añadir la siguiente línea a \lstinline|/etc/hosts|.

\begin{lstlisting}
192.168.100.119 Node03.srv.nis srv.nis Node03 srv
\end{lstlisting}

Esto redirecciona todas las peticiones del dominio del servidor a su ip. El
gestor de DNS configura de forma automática el registro en
\lstinline|/etc/resolv.conf|, quedando de la siguiente manera:

\begin{lstlisting}
# Dynamic resolv.conf(5) file for glibc resolver(3) generated by resolvconf(8)
#     DO NOT EDIT THIS FILE BY HAND -- YOUR CHANGES WILL BE OVERWRITTEN
nameserver 192.168.100.119
nameserver 8.8.8.8
search srv.nis
\end{lstlisting}

\subsubsection{NIS}\label{sec:nis}

NIS funciona para poder centralizar la autenticación de los clientes Linux.
\lstset{style=mystyle}
\begin{enumerate}
\item Instalar NIS, en terminal con permisos administrativos:

  \begin{lstlisting}[language=bash,style=mystyle]
apt -y install nis
  \end{lstlisting}

  Al finalizar aparecerá una pantalla de configuración donde se
  añadirá el dominio del servidor

  \begin{lstlisting}
NIS domain:

srv.nis______

    <ok>
\end{lstlisting}
  
\item Configurar como servidor maestro NIS

  Se tiene que modificar el archivo \lstinline|/etc/default/nis|

  \begin{lstlisting}[label={list:nis},caption=Modificación del archivo /etc/default/nis]
# Linea 6: Poner a NIS como servidor maestro
NISSERVER=master
\end{lstlisting}

  Adicionalmente en el mismo archivo de configuración, se puede
  configurar un rango de IPs que pueden hacer peticiones
  a este servicio

  \begin{lstlisting}
# Si se deja asi se le dara acceso a todo el mundo
0.0.0.0 0.0.0.0
# Si se configura asi se le dara acceso solo al rango deseado
192.168.100.0 192.168.100.255
\end{lstlisting}

  Reiniciamos el servicio nis para que se efectúen los cambios.

  \begin{lstlisting}[language=bash,style=mystyle]
systemctl restart nis
\end{lstlisting}

  
\item Aplicar la configuración al servicio

  Ejecutamos el siguiente comando

  \begin{lstlisting}
/usr/lib/yp/ypinit -m
\end{lstlisting}

  Si todo va bien se tiene que aparecer lo siguiente:

  \begin{lstlisting}
Node03.srv.nis has been set up as a NIS master server.

Now you can run ypinit -s Node03.srv.nis on all slave server.
\end{lstlisting}
  
\item Cada que se tenga  que añadir un nuevo usuario se
  tiene que actualizar la base de datos de NIS\@
  (este ya esta incluido en el script \lstinline|add_user.sh|).

  Se ejecuta el siguiente comando dentro del directorio
  \lstinline|/var/yp|

  \begin{lstlisting}
make
\end{lstlisting}
  
\end{enumerate}

\subsubsection{NFS}\label{sec:nfs}

NFS crea un sistema de archivos centralizados por redefined
\begin{enumerate}
\item Instalar el servidor nfs

  \begin{lstlisting}[language=bash]
apt -y install nfs-kernel-server
\end{lstlisting}

\item Configurar el dominio del servidor en el
  archivo \lstinline|/etc/idmapd.conf|

  \begin{lstlisting}[label={list:idmapd},  caption=Modificación del archivo /etc/idmap.conf]
# Linea 6: Aqui se descomenta y se agrega el dominio
Domain = srv.nis
\end{lstlisting}

\item Añadir la ruta de los directorios home que se van a
  compartir por NFS, esto es en el archivo \lstinline|/etc/exports|

  \begin{lstlisting}[label={list:exports},caption=Adición en el archivo /etc/exports]
/home 192.168.100.0/24(rw,no_root_squash,no_subtree_check) 
\end{lstlisting}

  \begin{itemize}
  \item \lstinline|/home| es la ruta donde se van a montar
    los directorios personales de los clientes.
  \item \lstinline|xx.xx.xx.xx/xx| Es la mascara del segmento que
    puede acceder a estos directorios por NFS.\@
  \item \lstinline|(..*)| Son las opciones de exports.
  \end{itemize}

  
\item Reiniciar el servicio para ver reflejados los cambios.

  \begin{lstlisting}
systemctl restart nfs-server
\end{lstlisting}
  
\end{enumerate}

\subsubsection{SAMBA AD DC}\label{sec:samba_addc}

SAMBA es una implementación del protocolo smb, a partir de su
versión 4 añade capacidades para crear y gestionar un controlador
de directorio activo (active directory) y kerberos, el cual es
compatible con la autenticación de red por de windows. Active
directory es una implementación del protocolo ldap y kerberos
es un protocolo de autenticación.

\begin{enumerate}
\item Instalar el protocolo para la sincronización de la hora.
  Es un requerimiento de kerberos para los miembros del dominio

  \begin{lstlisting}
apt install ntp
\end{lstlisting}

  \item Instalar los paquetes necesarios para el servidor de
    Samba 4 con AD DC

    \begin{lstlisting}
apt install samba smbclient attr winbind libpam-winbind libnss-winbind libpam-krb5 krb5-config krb5-user
\end{lstlisting}

  Mostrara una ventana de configuración que pedirá algunos parámetros

  \begin{enumerate}
  \item El primero es el del REALM o reino:

    \begin{lstlisting}
Reino predeterminado de la versión 5 de Kerberos:
SRV.NIS_________
    <Aceptar>
  \end{lstlisting}

  
   \item El siguiente es el nombre del host, el cual se usara el mismo
     que el reino pero en minúsculas

     \begin{lstlisting}
Servidores de Kerberos para su reino:
srv.nis______
    <Aceptar>
\end{lstlisting}

    \item La ultima ventana pedirá el nombre del host administrativo.
      Se pone el mismo que el del servidor

      \begin{lstlisting}
Servidor administrativo para su reino de Kerberos:
srv.nis_______
    <Aceptar>
\end{lstlisting}


  
    \end{enumerate}

  \item Creación del controlador de dominio.

    Se detienen los servicios antes de configurar esta parte.
\lstset{style=mystyle}
    \begin{lstlisting}[style=mystyle]
systemctl stop samba-ad-dc smbd nmbd winbind
systemctl disable samba-ad-dc smbd nmbd winbind
\end{lstlisting}

    Se elimina o se respalda el archivo de configuración de
    SAMBA por defecto
    \begin{lstlisting}
mv /etc/samba/smb.conf /etc/samba/smb.conf.org
\end{lstlisting}

    Se inicia la creación del controlador de forma interactiva, dotándole de compatibilidad con extensiones NIS RFC2307.

    \begin{lstlisting}
samba-tool domain provision --use-rfc2307 --interactive
\end{lstlisting}

    En la parte de Realm introducir el usado en este manual.

    \begin{lstlisting}
Realm: srv.nis
\end{lstlisting}

    En domain dejar el que esta por defecto, solo pulsar enter

    \begin{lstlisting}
Domain [SRV]:
\end{lstlisting}

    En Server Role dejar el que esta por defecto [dc]

    \begin{lstlisting}
Server Role (dc, member, standalon) [dc]:
    \end{lstlisting}

    DNS backend, dejar el que esta por defecto que es SAMBA\_INTERNAL

    \begin{lstlisting}
DNS backend (SAMBA_INTERNAL, BIND9_FLATFILE, BIND9_DLZ, NONE) [SAMBA_INTERNAL]:
\end{lstlisting}

    DNS fowarder IP address. Dejar la IP del servidor que en este caso es 192.168.100.119

    \begin{lstlisting}
DNS forwarder IP address (write 'none' to disable forwarding) [127.0.0.1]: 192.168.100.119
\end{lstlisting}

    Administrator password: Esta es la contraseña de administrador, poner una que sea mayor a 8 caracteres con una mayúscula y un dígito

    \begin{lstlisting}
Administrator password:
Retype password:
\end{lstlisting}

    Si todo sale bien mostrara los datos con controlador de dominio

    \begin{lstlisting}
Server Role:           active directory domain controller
Hostname:              Node03
NetBIOS Domain:        SRV
DNS Domain:            srv.nis
DOMAIN SID:            S-1-5-21-3772837808-1505251784-1375148484
\end{lstlisting}

    Iniciar la familia de los demonios del samba-ad-dc

    \begin{lstlisting}
systemctl unmask samba-ad-dc
systemctl start samba-ad-dc
systemctl enable samba-ad-dc
\end{lstlisting}

    
  \item Probar la configuración

    Verificar el nivel de dominio

    \begin{lstlisting}
samba-tool domain level show
\end{lstlisting}

    Si todo sale bien debe mostrar lo siguiente
\lstset{style=nonestyle}
    \begin{lstlisting}
Domain and forest function level for domain 'DC=srv,DC=nis'

Forest function level: (Windows) 2008 R2
Domain function level: (Windows) 2008 R2
Lowest function level of a DC: (Windows) 2008 R2
\end{lstlisting}

    Verificar el servidor de archivos. \lstinline|netlogon|
    y \lstinline|sysvol|
\lstset{style=mystyle}
    \begin{lstlisting}
smbclient -L localhost -U%
\end{lstlisting}

    Debe mostrar lo siguiente:
\lstset{style=nonestyle}
    \begin{lstlisting}
       Sharename       Type      Comment
        ---------       ----      -------
        homes           Disk      Home Directories
        netlogon        Disk
        sysvol          Disk
        IPC$            IPC       IPC Service (Samba 4.9.5-Debian)
Reconnecting with SMB1 for workgroup listing.

        Server               Comment
        ---------            -------

        Workgroup            Master
        ---------            -------
        WORKGROUP            NODE03
        WORKSOMCH            VENGANZASS
\end{lstlisting}

    	En el caso anterior se mostró los directorios configurados y 
    		los workgroups existentes de otras maquinas Windows en la red.
    	
    	Verificar la autenticación usando el usuario de administrador del dominio.
    	
    	\begin{lstlisting}
smbclient //localhost/netlogon -UAdministrator -c 'ls'
\end{lstlisting}

    	Si todo sale bien debe mostrar lo siguiente:
    	
    	\begin{lstlisting}
Enter SRV\Administrator's password:
  .                                   D        0  Sun May 10 20:07:09 2020
  ..                                  D        0  Sun May 10 20:07:12 2020

                19478160 blocks of size 1024. 17106040 blocks available

    	\end{lstlisting}
    	
    	\item Verificar los registros de DNS. Importante que si los muestre ya 
    		que sin estos Windows no sera capaz de detectar el dominio
    		
    		SRV de ldap usando TCP
\lstset{style=mystyle}    		
    		\begin{lstlisting}
host -t SRV _ldap._tcp.srv.nis
\end{lstlisting}
    	
		SRV de kerberos usando UDP


		\begin{lstlisting}
host -t SRV _kerberos._udp.srv.nis
\end{lstlisting}

		A del dominio
		
		\begin{lstlisting}
host -t A Node03.srv.nis
\end{lstlisting}
		
		\item Si todo salio bien entonces el servidor ya esta correctamente configurado
		
		A veces hay que abrir los puertos en el firewall en caso de tener problemas


    	
    	
    
\end{enumerate}


\end{document}
