\documentclass[../main.tex]{subfiles}

\begin{document}
\subsection{Red Emulada}\label{sec:red_emu}

\begin{table}[H]
  \centering
  \begin{tabular}{rl}
    
    Segmento:&$192.168.100.0/24$\\
    Puerta de enlace:&$192.168.100.1$\\
    Broadcast:&$192.168.100.255$\\
    Dominio:&\texttt{srv.nis}\\
  \end{tabular}
\end{table}

\subsection{Servidor}\label{sec:servidor}

\subsubsection{Servidor Linux (VM)}\label{sec:slvm}


\begin{table}[H]
  \centering
  \begin{tabular}{rl}
    
    Hostname: &Node03\\
    Sistema Operativo: & Debian 10 \textit{Buster}\\
  \end{tabular}
\end{table}

\subsubsection{Tarjeta de Red}\label{sec:tr}

\begin{table}[H]
  \centering
  \begin{tabular}{rl}
    
    IP:&$192.16.100.119/24$\\
    DNS:&$192.168.100.119 8.8.8.8$\\
  \end{tabular}
\end{table}

\subsection{Configuración}\label{sec:serv_conf}

\subsubsection{Configurar la tarjeta de red}\label{sec:conf_tr}

Se tiene que configurar la tarjeta de red para que adquiera su DNS y
ip estática:

\begin{itemize}
  \item En este caso la interfaz de red es \texttt{ens33}, donde
        este nombre puede variar.
  \item Se tiene que modificar el archivo \texttt{/etc/network/interfaces}
        y añadir la siguiente configuración:

        \begin{listing}[H]
\begin{minted}{linux-config}
auto ens33
allow-hotplug ens33
iface ens33 inet static
    address 192.168.100.119
    netmask 255.255.255.0
    network 192.168.100.0
    broadcast 102.168.100.255
    gateway 192.168.100.1
    dns-nameservers 192.168.100.119 8.8.8.8
    dns-search srv.nis
\end{minted}
    \caption{Archivo /etc/netctl/interfaces.}
\label{listing:interfaces}
    \end{listing}
\end{itemize}

\paragraph{Asignar Dominio}\ \\
Se debe de añadir la siguiente línea a \texttt{/etc/hosts}.

\begin{listing}[H]
\begin{minted}{linux-config}
192.168.100.119 Node03.srv.nis srv.nis Node03 srv
\end{minted}
\end{listing}

Esto redirecciona todas las peticiones del dominio del servidor a su ip. El
gestor de DNS configura de forma automática el registro en
\texttt{/etc/resolv.conf}, quedando de la siguiente manera:

\begin{listing}[H]
\begin{minted}{linux-config}
# Dynamic resolv.conf(5) file for glibc resolver(3) generated by resolvconf(8)
#     DO NOT EDIT THIS FILE BY HAND -- YOUR CHANGES WILL BE OVERWRITTEN
nameserver 192.168.100.119
nameserver 8.8.8.8
search srv.nis
\end{minted}
\end{listing}

\subsubsection{NIS}\label{sec:nis}

NIS funciona para poder centralizar la autenticación de los clientes Linux.

\begin{enumerate}
  \item Instalar NIS, en terminal con permisos administrativos:

        \begin{listing}[H]
\begin{minted}{shell-session}
$ apt -y install nis
\end{minted}
\end{listing}

        Al finalizar aparecerá una pantalla de configuración donde se
        añadirá el dominio del servidor

        \begin{listing}[H]
\begin{minted}{linux-config}
NIS domain:

srv.nis______

    <ok>
\end{minted}
\end{listing}

  \item Configurar como servidor maestro NIS

        Se tiene que modificar el archivo \texttt{/etc/default/nis}

        \begin{listing}[H]
\begin{minted}{linux-config}
# Linea 6: Poner a NIS como servidor maestro
NISSERVER=master
\end{minted}
    \caption{Modificación del archivo /etc/default/nis}
    \label{listing:nis}
\end{listing}

        Adicionalmente en el mismo archivo de configuración, se puede
        configurar un rango de IPs que pueden hacer peticiones
        a este servicio

        \begin{listing}[H]
\begin{minted}{linux-config}
# Si se deja asi se le dara acceso a todo el mundo
0.0.0.0 0.0.0.0
# Si se configura asi se le dara acceso solo al rango deseado
192.168.100.0 192.168.100.255
\end{minted}
\end{listing}

        Reiniciamos el servicio nis para que se efectúen los cambios.

        \begin{listing}[H]
\begin{minted}{shell-session}
$ systemctl restart nis
\end{minted}
\end{listing}


  \item Aplicar la configuración al servicio

        Ejecutamos el siguiente comando

        \begin{listing}[H]
\begin{minted}{shell-session}
$ /usr/lib/yp/ypinit -m
\end{minted}
\end{listing}

        Si todo va bien se tiene que aparecer lo siguiente:

        \begin{listing}[H]
\begin{minted}{shell-session}
Node03.srv.nis has been set up as a NIS master server.

Now you can run ypinit -s Node03.srv.nis on all slave server.
\end{minted}
\end{listing}

  \item Cada que se tenga  que añadir un nuevo usuario se
        tiene que actualizar la base de datos de NIS\@
        (este ya esta incluido en el script \texttt{add\_user.sh}).

        Se ejecuta el siguiente comando dentro del directorio
        \texttt{/var/yp}

        \begin{listing}[H]
\begin{minted}{shell-session}
$ make
\end{minted}
\end{listing}

\end{enumerate}

\subsubsection{NFS}\label{sec:nfs}

NFS crea un sistema de archivos centralizados por redefined
\begin{enumerate}
  \item Instalar el servidor nfs

        \begin{listing}[H]
\begin{minted}{shell-session}
$ apt -y install nfs-kernel-server
\end{minted}
\end{listing}

  \item Configurar el dominio del servidor en el
        archivo \texttt{/etc/idmapd.conf}

        \begin{listing}[H]
\begin{minted}{linux-config}
# Linea 6: Aqui se descomenta y se agrega el dominio
Domain = srv.nis
\end{minted}
    \caption{Modificación del archivo /etc/idmap.conf}
    \label{listing:idmapd}
\end{listing}

  \item Añadir la ruta de los directorios home que se van a
        compartir por NFS, esto es en el archivo \texttt{/etc/exports}

        \begin{listing}[H]
\begin{minted}{linux-config}
/home 192.168.100.0/24(rw,no_root_squash,no_subtree_check) 
\end{minted}
    \caption{Adición en el archivo /etc/exports}
    \label{listing:exports}
\end{listing}

        \begin{itemize}
          \item \texttt{/home} es la ruta donde se van a montar
                los directorios personales de los clientes.
          \item \texttt{xx.xx.xx.xx/xx} Es la mascara del segmento que
                puede acceder a estos directorios por NFS.\@
          \item \texttt{(..*)} Son las opciones de exports.
        \end{itemize}


  \item Reiniciar el servicio para ver reflejados los cambios.

        \begin{listing}[H]
\begin{minted}{shell-session}
$ systemctl restart nfs-server
\end{minted}
\end{listing}

\end{enumerate}

\subsubsection{SAMBA AD DC}\label{sec:samba_addc}

SAMBA es una implementación del protocolo smb, a partir de su
versión 4 añade capacidades para crear y gestionar un controlador
de directorio activo (active directory) y kerberos, el cual es
compatible con la autenticación de red por de windows. Active
directory es una implementación del protocolo ldap y kerberos
es un protocolo de autenticación.

\begin{enumerate}
  \item Instalar el protocolo para la sincronización de la hora.
        Es un requerimiento de kerberos para los miembros del dominio

        \begin{listing}[H]
\begin{minted}{shell-session}
$ apt install ntp
\end{minted}
\end{listing}

  \item Instalar los paquetes necesarios para el servidor de
        Samba 4 con AD DC

        \begin{listing}[H]
\begin{minted}{shell-session}
$ apt install samba smbclient attr winbind libpam-winbind libnss-winbind\
    libpam-krb5 krb5-config krb5-user
\end{minted}
\end{listing}

        Mostrara una ventana de configuración que pedirá algunos parámetros

        \begin{enumerate}
          \item El primero es el del REALM o reino:

                \begin{listing}[H]
\begin{minted}{shell-session}
Reino predeterminado de la versión 5 de Kerberos:
SRV.NIS_________
    <Aceptar>
\end{minted}
\end{listing}


          \item El siguiente es el nombre del host, el cual se usara el mismo
                que el reino pero en minúsculas

                \begin{listing}[H]
\begin{minted}{shell-session}
Servidores de Kerberos para su reino:
srv.nis______
    <Aceptar>
\end{minted}
\end{listing}

          \item La ultima ventana pedirá el nombre del host administrativo.
                Se pone el mismo que el del servidor

                \begin{listing}[H]
\begin{minted}{shell-session}
Servidor administrativo para su reino de Kerberos:
srv.nis_______
    <Aceptar>
\end{minted}
\end{listing}



        \end{enumerate}

  \item Creación del controlador de dominio.

        Se detienen los servicios antes de configurar esta parte.

        \begin{listing}[H]
\begin{minted}{shell-session}
$ systemctl stop samba-ad-dc smbd nmbd winbind
$ systemctl disable samba-ad-dc smbd nmbd winbind
\end{minted}
\end{listing}

        Se elimina o se respalda el archivo de configuración de
        SAMBA por defecto
        \begin{listing}[H]
\begin{minted}{shell-session}
$ mv /etc/samba/smb.conf /etc/samba/smb.conf.org
\end{minted}
\end{listing}

        Se inicia la creación del controlador de forma interactiva, dotándole de compatibilidad con extensiones NIS RFC2307.

        \begin{listing}[H]
\begin{minted}{shell-session}
$ samba-tool domain provision --use-rfc2307 --interactive
\end{minted}
\end{listing}

        En la parte de Realm introducir el usado en este manual.

        \begin{listing}[H]
\begin{minted}{shell-session}
Realm: srv.nis
\end{minted}
\end{listing}

        En domain dejar el que esta por defecto, solo pulsar enter

        \begin{listing}[H]
\begin{minted}{shell-session}
Domain [SRV]:
\end{minted}
\end{listing}

        En Server Role dejar el que esta por defecto [dc]

        \begin{listing}[H]
\begin{minted}{shell-session}
Server Role (dc, member, standalon) [dc]:
\end{minted}
\end{listing}

        DNS backend, dejar el que esta por defecto que es SAMBA\_INTERNAL

        \begin{listing}[H]
\begin{minted}{shell-session}
DNS backend (SAMBA_INTERNAL, BIND9_FLATFILE, BIND9_DLZ, NONE) [SAMBA_INTERNAL]:
\end{minted}
\end{listing}

        DNS fowarder IP address. Dejar la IP del servidor que en este caso es 192.168.100.119

        \begin{listing}[H]
\begin{minted}{shell-session}
DNS forwarder IP address (write 'none' to disable forwarding) [127.0.0.1]: 192.168.100.119
\end{minted}
\end{listing}

        Administrator password: Esta es la contraseña de administrador, poner una que sea mayor a 8 caracteres con una mayúscula y un dígito

        \begin{listing}[H]
\begin{minted}{shell-session}
Administrator password:
Retype password:
\end{minted}
\end{listing}

        Si todo sale bien mostrara los datos con controlador de dominio

        \begin{listing}[H]
\begin{minted}{shell-session}
Server Role:           active directory domain controller
Hostname:              Node03
NetBIOS Domain:        SRV
DNS Domain:            srv.nis
DOMAIN SID:            S-1-5-21-3772837808-1505251784-1375148484
\end{minted}
\end{listing}

        Iniciar la familia de los demonios del samba-ad-dc

        \begin{listing}[H]
\begin{minted}{shell-session}
$ systemctl unmask samba-ad-dc
$ systemctl start samba-ad-dc
$ systemctl enable samba-ad-dc
\end{minted}
\end{listing}


  \item Probar la configuración

        Verificar el nivel de dominio

        \begin{listing}[H]
\begin{minted}{shell-session}
$ samba-tool domain level show
\end{minted}
\end{listing}

        Si todo sale bien debe mostrar lo siguiente

        \begin{listing}[H]
\begin{minted}{shell-session}
Domain and forest function level for domain 'DC=srv,DC=nis'

Forest function level: (Windows) 2008 R2
Domain function level: (Windows) 2008 R2
Lowest function level of a DC: (Windows) 2008 R2
\end{minted}
\end{listing}

        Verificar el servidor de archivos. \texttt{netlogon}
        y \texttt{sysvol}

        \begin{listing}[H]
\begin{minted}{shell-session}
$ smbclient -L localhost -U%
\end{minted}
\end{listing}

        Debe mostrar lo siguiente:

        \begin{listing}[H]
\begin{minted}{shell-session}
       Sharename       Type      Comment
        ---------       ----      -------
        homes           Disk      Home Directories
        netlogon        Disk
        sysvol          Disk
        IPC$            IPC       IPC Service (Samba 4.9.5-Debian)
Reconnecting with SMB1 for workgroup listing.

        Server               Comment
        ---------            -------

        Workgroup            Master
        ---------            -------
        WORKGROUP            NODE03
        WORKSOMCH            VENGANZASS
\end{minted}
\end{listing}

        En el caso anterior se mostró los directorios configurados y
    		los workgroups existentes de otras maquinas Windows en la red.

        Verificar la autenticación usando el usuario de administrador del dominio.

        \begin{listing}[H]
\begin{minted}{shell-session}
$ smbclient //localhost/netlogon -UAdministrator -c 'ls'
\end{minted}
\end{listing}

        Si todo sale bien debe mostrar lo siguiente:

        \begin{listing}[H]
\begin{minted}{shell-session}
Enter SRV\Administrator's password:
  .                                   D        0  Sun May 10 20:07:09 2020
  ..                                  D        0  Sun May 10 20:07:12 2020

                19478160 blocks of size 1024. 17106040 blocks available

\end{minted}
\end{listing}

  \item Verificar los registros de DNS. Importante que si los muestre ya 
    		que sin estos Windows no sera capaz de detectar el dominio
    		
    		SRV de ldap usando TCP

    		\begin{listing}[H]
\begin{minted}{shell-session}
$ host -t SRV _ldap._tcp.srv.nis
\end{minted}
\end{listing}

        SRV de kerberos usando UDP


        \begin{listing}[H]
\begin{minted}{shell-session}
$ host -t SRV _kerberos._udp.srv.nis
\end{minted}
\end{listing}

        A del dominio

        \begin{listing}[H]
\begin{minted}{shell-session}
$ host -t A Node03.srv.nis
\end{minted}
\end{listing}

  \item Si todo salio bien entonces el servidor ya esta correctamente configurado

        A veces hay que abrir los puertos en el firewall en caso de tener problemas





\end{enumerate}


\end{document}
