\newglossaryentry{SAMBA}
{
  name=SAMBA,
  description={\ \ \ \
Es una implementación libre del protocolo de archivos
    compartidos de Microsoft Windows para sistemas de tipo UNIX}
}
\newglossaryentry{ldap}
{
  name=LDAP,
  description={\ \ \ \
El protocolo ligero de acceso a directorios hace referencia a un
    protocolo a nivel de aplicación que permite el acceso a un servicio de
    directorio ordenado y distribuido para buscar diversa información en un
    entorno de red}
}

\newglossaryentry{kerberos}
{
  name=Kerberos,
  description={\ \ \ \
Es un protocolo de autenticación de redes de ordenador creado
    por el MIT que permite a dos ordenadores en una red insegura demostrar
    su identidad mutuamente de manera segura}
}

\newglossaryentry{nfs}
{
  name=NFS,
  description={\ \ \ \
Network File System es un
    protocolo de nivel de aplicación, según el Modelo OSI.~Es utilizado para
    sistemas de archivos distribuido en un entorno de red de computadoras de
    área local. Posibilita que distintos sistemas conectados a una misma red
    accedan a ficheros remotos}
}

\newglossaryentry{autofs}
{
  name=AutoFS,
  description={\ \ \ \
Es un servicio por parte del cliente que monta
    automáticamente el sistema de archivos adecuado}
}

\newglossaryentry{fstab}
{
  name=fstab,
  description={\ \ \ \
Es un fichero que se encuentra comúnmente en
    sistemas Unix (en el directorio \texttt{/etc/}) como parte de la
    configuración del sistema. Lo más destacado de este fichero es la lista de
    discos y particiones disponibles. En ella se indica como montar cada
    dispositivo y qué configuración utilizar}
}

\newglossaryentry{systemd}
{
  name=systemd,
  description={\ \ \ \
Es un conjunto de demonios o daemons de
    administración de sistema, bibliotecas y herramientas diseñados como una
    plataforma de administración y configuración central para interactuar con
    el núcleo del Sistema operativo GNU/Linux}
}

\newglossaryentry{ntp}
{
  name=NTP,
  description={\ \ \ \
Network Time Protocol es un protocolo de Internet para
    sincronizar los relojes de los sistemas informáticos a través del
    enrutamiento de paquetes en redes con latencia variable
  }
}





\newacronym{ntp}{NTP}{Network Time Protocol}
\newacronym{nfs}{NFS}{Network File System}
\newacronym{nis}{NIS}{Network Information Service}
\newacronym{smb}{SMB}{Server Message Block}
